\newglossaryentry{valleyFold}{%
  name={Valley Fold},
  description={The fold that results after folding one part of the paper over the other. After unfolding you roughly see a V-shape}
}

\newglossaryentry{mountainFold}{%
  name={Mountain Fold},
  description={The fold that results after folding one part of the paper behind the other. After unfolding you roughly see an A-shape}
}

\newglossaryentry{x-rayLine}{%
  name={X-Ray Line},
  description={An X-Ray line indicates a fold or edge that is hidden behind a layer of paper}
}

\newglossaryentry{arrowsOfMotion}{%
  name={Arrows of Motion},
  description={Arrows of Motion indicate where the paper folded towards}
}

\newglossaryentry{arrowsOfAction}{%
  name={Arrows of Action},
  description={Arrows of Action indicate an action performed on the paper}
}

\newglossaryentry{rabbitEar}{%
  name={Rabbit Ear},
  description={A Rabbit Ear is a folding sequence, that narrows the paper and creates a new flap. This is done by creating three folds at the bisectors of a triangle}
}

\newglossaryentry{origami}{%
  name={Origami},
  description={(jpn: \emph{ori = folding} and \emph{kami = paper}) is the art of folding paper into models of animals, people or other objects}
}

\newglossaryentry{origrammer}{%
  name={Origrammer},
  description={Origrammer is the name of the diagramming program developed within the scope of this project}
}

\newglossaryentry{diagram}{%
  name={diagram},
  description={Origami diagrams give folding instructions for origami models. They usually consist of a representation of the model per folding step, the folding instruction in symbol form, as well as the folding instruction in text form.}
}