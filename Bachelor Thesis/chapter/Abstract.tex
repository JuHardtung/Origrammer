%!TEX root = ../Hardtung_BA_SoSe20.tex

\section{Abstract}
\label{sec:abstract}

The \gls{origrammer} is a desktop application for creating origami instructions, also called origami \gls{diagram}s. The goal of this program was to maximize the efficiency of creating these diagrams, as the process is slow and tedious when creating them by hand.
In this paper the current state of this program was evaluated using the 10 usability heuristics by Jakob Nielsen \cite{10usability_heuristics} with particular focus on flaws that reduce efficiency and speed of the diagram creation. With the found flaws as a basis the Origrammer was further developed and improved.

A more flexible navigation through the folding steps with editing functions and step preview pictures was added. To further maximize efficiency, the whole system of rendering the paper was reworked as well. The Origrammer can now virtually fold paper, without the need to place every single edge or crease line by hand. Simple \gls{mountainFold}s and \gls{valleyFold}s can be carried out and the paper folds automatically. %Though this system is still in the beginning stages and needs further research and development.

Various additional improvements were introduced, like folding presets for similar beginnings of an origami model, a reworked arrow system that is more flexible and improves on several issues, more line inpout variants, and a dynamic preview when inputting lines, arrows, or symbols.