%!TEX root = ../Hardtung_BA_SoSe20.tex

\section{Outlook}
\label{sec:prospect}

The automation of processes in the Origrammer has changed the workflow of creating diagrams as a whole. Instead of placing lines, arrows, and symbols one by one, the user can simply fold the paper with two mouse clicks, thereby folding the paper automatically. Furthermore, the correct folding arrow is placed without any additional user input and the following step is started automatically.

With the use of the step navigation, the user can navigate more efficiently and flexible. Steps can be moved, duplicated, deleted and created freely. Moreover, the preview images of said steps shows the user where he is in the diagram.

Another important improvement is made for the input of lines, arrows, and symbols. During the input process dynamic previews are being shown to guide the user which input sequences have to be strictly followed . This will help especially novices that don't necessarily know the specific input structure of all objects.\\
\newline
Future development could improve on the virtual folding, especially for more complex folding actions like reverse folds, crimps and pleats, and opened/mixed/closed sinks. Another issue of the current folding system is the inability to display curves and 3d-shaped parts of the origami model. This would be especially important during the shaping process at the end of folding a model.

Additionally, the user interface still contains several usability issues, even though most of them were already discovered during the heuristic evaluation. Consequently, the user interface will have to be reworked and improved in the future. At this later point the involvement of real users appears valuable to maximise usability improvements.