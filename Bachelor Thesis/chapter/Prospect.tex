%!TEX root = ../Hardtung_BA_SoSe20.tex

\section{Prospect}
\label{sec:prospect}

The automation of processes in the Origrammer has changed the workflow of creating diagrams as a whole. Instead of placing lines, arrows, and symbols one by one, the user can simply fold the paper with two mouse clicks and the paper is folded automatically. Furthermore, the correct folding arrow is placed without any additional user input and the following step is started automatically.

With the use of the step navigation, the user can navigate more efficiently and flexible. Steps can be moved, duplicated, deleted and created freely and the preview images of said steps shows the user where he is in the diagram.

Another important improvement was made for the input of lines, arrows, and symbols. During the input process dynamic previews are being shown to guide the user on what input sequences have to be strictly followed . This will help especially beginners that don't necessarily know the specific input structure of all objects.\\
\newline
Further development in the future could improve on the virtual folding, especially for more complex folding actions like reverse folds, crimps and pleats, and opened/mixed/closed sinks. Another flaw of the current folding system is the inability to display curves and 3d-shaped parts of the origami model. This would be especially important during the shaping process at the end of folding a model.

The user interface also still contains several usability issues, though most of which were already discovered during the heuristic evaluation. Consequently, the user interface will have to be reworked or improved upon in the future. This is also the point where real users should be involved during additional evaluation in order to maximise the usability improvements.